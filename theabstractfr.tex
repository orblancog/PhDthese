\begin{abstract}
L'exploration d'une nouvelle physique \`a l'\'echelle d'\'energie des \og~Tera electron Volt~\fg~(TeV) n\'ecessite de collisionner des leptons dans de grands acc\'el\'erateurs lin\'eaires \`a grande luminosit\'e. Ils permettent des mesures de precision avec une statistique suffisante pour analyser les interactions entre les particules. Afin d'atteindre une grande luminosit\'e, les collisionneurs lin\'eaires requiert une taille de faisceau \`a l'echelle nanom\'etrique au Point d'Interaction~(IP).\par
Parmi les multiples effets participant \`a la degradation de la luminosit\'e, la correction de la chromaticit\'e, l'effet du rayonnement synchrotronique et la correction des erreurs dans la ligne sont parmi les trois effets \`a ma\^itriser afin de r\'eduire la taille du faisceau dans la Section Finale de Focalisation~(FFS).\par
Actuellement, les projets de collisionneur lin\'eaire dont le~\og~International Linear Collider~\fg~(ILC)~\cite{ILCdes} et le~\og~Compact Linear Collider~\fg~(CLIC)~\cite{CLICdes} sont \'etudi\'es en utilisant des sch\'emas de correction de la chromaticit\'e, pouvant \^etre locale, ou non locale.\par
Cette these propose un nouveau sch\'ema de correction de la chromaticit\`e que l'on appelera ``~non-entrelac\'e~'', appliqu\'e ici au projet CLIC. Lors de l'implementation de cette nouvelle methode, il a \'et\'e mis en evidence que le probl\'eme principal est la dispersion de deuxi\`eme ordre au Doublet Final~(FD), qui traverse un sextupole utilis\'e pour annuler les composantes g\'eometriques restantes. Cela pourrait \^etre r\'esolu en annulant la dispersion de deuxi\`eme ordre et sa d\'eriv\'ee en amont du FD.\par
L'effet du rayonnement peut \^etre evalu\'e par m\'ethode de tracking des particules ou par des approximations analytiques lors de la conception de la maille. Afin d'inclure ces effets du rayonnement et les param\'etres optiques de la ligne pendant la conception et le processus d'optimisation, l'effet Oide~\cite{Oide} et le rayonnement d\^u aux aimants dipolaires~\cite{Sands} ont ete etudi\'es.\par
Le r\'esultat analytique du rayonnement synchrotronique dans les aimants dipolaires~\cite{Sands} fut generalis\'e dans les cas avec alpha et dispersion non-nulles \`a l'IP. Cette g\'en\'eralisation est utilis\'ee pour am\'eliorer le code de simulation PLACET~\cite{Placet} en le comparant avec la solution particuli\`ere pour un aimant dipolaire et un aimant dipolaire plus une section droite.\par
Le rayonnement dans les aimants quadripolaires finaux imposent une limite \`a la taille verticale minimale du faiceau, connu comme l'effet Oide. Celui-ci est uniquement important \`a 3~TeV, donc deux possibilit\'es sont explor\'ees pour att\'enuer sa contribution dans la taille du faisceau: doubler la longueur et r\'eduire le gradient du derni\`er quadripole (QD0), ou integrer une paire d'aimants octupolaires, un en amont et un en aval du QD0.\par
Une partie des exigences du FFS pour les nouveaux collisionneurs lin\'eaire \`a leptons, en particuli\`ere pour ILC, est test\'ee exp\'erimentalement dans l'\og~Accelerator Test Facility~\fg~(~ATF~). La r\'eduction de la taille du faisceau d'\'electrons en utilisant le sch\'ema local de correction de la chromaticit\'e est explor\'ee dans une extension de la ligne originale, appell\'e ATF2, o\'u deux buts furent fix\'es~: (~{\textbf{but 1}}~) atteindre 37~nm de taille verticale du faisceau \`a l'IP, et (~{\textbf{but 2}}~) stabiliser de l'ordre du nanom\`etre la position verticale du faisceau \`a l'IP.\par
Depuis 2014, une taille de 44~nm avec un nombre de particules d'environ $0.1\times10^{10}$ par paquet est atteint de mani\`ere reguli\`ere. Des cavit\'es radio-frequence seront utilis\'ees pour la stabilisation du faisceau, et \'egalement pour d\'etecter le d\'eplacement/les fluctuations du faisceau au dehors la marge
tolerable pour le syst\'eme de mesure, ainsi que des erreurs non detect\'ees dans l'optique.\par
Un set de trois cavit\'es (~IPA, IPB et IPC~) furent install\'ees et sont utilis\'ees pour mesurer la trajectoire du faiceau dans la r\'egion de l'IP, fournissant ainsi des informations pour reconstruire la position et l'angle \`a l'IP. Les specifications pour l'optique nominale d'ATF2, i.e. 1~nm de r\'esolution sur 10~$\mu$m de gamme dynamique \`a un nombre de particules de $1.0\times10^{10}$ par paquet, n'ont pas encore \'et\'e atteint.\par
La meilleur r\'esolution atteinte jusqu'ici correspond \`a 50~nm pour $0.4\times10^{10}$ particules par paquet, o\`u le bruit de l'\'el\'ectronique impose une limite de 10~nm par cavit\'e sur la r\'esolution. La gamme dynamique est de 10~$\mu$m \`a $0.4\times10^{10}$ particules par paquet et 10~dB d'attenuation du signal des cavit\'es, n\'ec\'essitant de mettre l'\'electronique \`a niveau. La calibration des cavit\'es dans le plan vertical indique une lin\'earit\'e dans les 5\% sur deux ordres de magnitude d'att\'enuation du signal. L'integration de ces cavit\'es aux instruments r\'eguliers de r\'eglage d'ATF est en cours. Le test du syst\`eme d'asservissement pour stabiliser le faisceau a atteint une r\'eduction de la fluctuation jusqu'a 67~nm, compatible avec la r\'esolution des cavit\'es.\par
Deux am\'eliorations ont \'et\'e faites sur le syst\'eme apr\`es ces \'etudes. En premi\`ere lieu, les plans horizontal et vertical pourront \^{e}tre analys\'es simultanement. En second lieu, des filtres furent integr\'es au syst\'eme pour r\'eduire l'effet de la difference de frequence dans le processus de~\og~down-mixing~\fg~des signaux.
\end{abstract}
