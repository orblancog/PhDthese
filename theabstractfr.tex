\begin{abstract}
L'exploration de la nouvelle physique avec des measures de precision, \`a l'\'echelle d'\'energie des \og~Tera electron Volt~\fg~(TeV), a besoin de la collision des leptons dans collisionneurs de grand luminosit\'e donnant de la statistique suffisante pour l'analyse des interactions entre les particules. Afin d'atteindre cette luminosit\'e, les collisionneurs lin\'eaires requis tailles du faisceau nanom\'etriques au Point d'Interaction~(IP).\par
En plus de plusieurs effets qui reduisent la luminosit\'e, il y a trois sujets principaux \`a ma\^{i}triser pour r\'eduire la taille du faisceau dans la Section Finale de Focalisation (FFS)~: la correction de la chromaticit\'e, l'effet du rayonnement synchrotronique et la correction des erreurs dans la ligne.\par
Les projets de collisionneur lin\'eaire actuelle, le \og International Linear Collider \fg (ILC)~\cite{ILCdes} et le \og Compact Linear Collider \fg (CLIC)~\cite{CLICdes} sont con\c cues en utilisant le sch\'ema de correction de la chromaticit\'e locale et non-locale. Un nouvel sch\'ema, appell\'e non-entrelac\'e, est propos\'e en plus des deux derni\`eres pour CLIC. La ligne est con\c cue et diagnostiqu\'ee, o\`u le probl\`eme principal dans l'\'etat actuel est la dispersion de deuxi\`eme ordre dans la r\'egion du Doublet Final~(FD) qui traverse un sextupole \`a grand force utilis\'e pour annuler les composants g\'eometriques restants. Cela pourrait \^{e}tre r\'esolu en annulant la dispersion de deuxi\`eme ordre et sa d\'eriv\'ee en amont du FD.\par
L'effet du rayonnement peut \^{e}tre evalu\'e par m\'ethode de tra\c cage de particules ou par des approximations analytiques pendant la p\'eriode de conception de la lattice. Afin d'inclure ces \`effets du rayonnement et les param\'etres optiques de la ligne, pendant la conception et le processus d'optimisation, deux phenom\`enes de rayonnement sont revue~: l'effet Oide~\cite{Oide} et le rayonnement \`a cause des aimants dipolaires~\cite{Sands}.\par
Le r\'esultat analytique du rayonnement dans les aimants dipolaires~\cite{Sands} fut generalis\'e dans les cas avec alpha et dispersion non-nulles \`a l'IP, requis pendant la conception de la ligne et le processus d'optimisation. La solution particuli\`ere pour un aimant dipolaire et un aimant dipolaire plus une section droite sont compar\'es avec le code de simulation PLACET~\cite{Placet}, r\'esultant en l'am\'ellioration du code de tra\c cage.\par
Sur l'effet Oide, le rayonnement dans les aimants quadripolaires finaux imposent une limite \`a la taille verticale minimale du faiceau. Seulement pour CLIC 3~TeV cette limite est r\'elevant, donc deux possibilit\'es sont explor\'ees pour att\'enuer sa contribution \`a la taille~: doubler la longueur et r\'eduire le gradient de QD0, ou l'integration d'un paire d'aimants octupolaires, un en amont et un en aval du QD0.\par
Une partie des exigences du FFS pour les nouveaux collisionneurs lin\'eaire, en particuli\`ere ILC, sont test\'es dans le \og  Accelerator Test Facility \fg (ATF). La r\'eduction de la taille du faisceau en utilisant la correction chromaticit\'e locale est explor\'ee dans une extension de la ligne originale, appell\'e ATF2, o\'u deux buts furent fix\'es~: ({\textbf{but 1}}) atteindre 37~nm de taille verticale du faisceau \`a l'IP, et ({\textbf{but 2}}) la stabilisation de la position verticale du faisceau \`a l'IP de quelques nanom\`etres. Une taille de 44~nm pour un nombre de particules d'environ~$0.1\times10^{10}$ par paquet est atteint de mani\`ere reguli\`ere depuis 2014.\par
Un set de trois cavit\'es (IPA, IPB et IPC), deux en amont et une en aval de l'IP nominal, furent install\'ees et sont utilis\'ees pour mesurer la trajectoire du faiceau dans la r\'egion de l'IP, ainsi fournissant informations suffisantes pour reconstruire la position et l'angle \`a l'IP. Ces cavit\'es seront utilis\'ees pour la stabilisation du faisceau et pourraient d\'etecter la d\'erive/les fluctuations du faisceau au dehors la marge tolerable et des erreurs non detect\'ees dans l'optique affectant la mesure de la taille du faisceau.\par
Les specifications requis pour l'optique nominale d'ATF2, 1~nm de r\'esolution sur 10~$\mu$m de gamme dynamique \`a une nombre de particules de $1.0\times10^{10}$ par paquet, n'ont pas encore \'et\'e atteint. Les r\'esultat des \'etudes de la calibration des cavit\'es dans le plan vertical montrent une meilleur lin\'earit\'e que 5\% dans deux ordres de magnitude de attenuation du signal. La r\'esolution minimale atteinte est juste en dessous 50~nm \`a ~$0.4\times10^{10}$ particules par paquet, o\`u le bruit de l'\'el\'ectronique impose une limite de 10~nm par cavit\'e sur la r\'esolution. La gamme dynamique est 10~$\mu$m \`a 10~dB d'attenuation et $0.4\times10^{10}$~particules par paquet, en indiquant la n\'ec\'essit\'e de mettre au niveau l'\'electronique. L'integration des cavit\'es aux instruments de r\'eglage d'ATF est en cours. Pourtant, le syst\`eme d'asservissement pour stabiliser le faisceau a \'et\'e test\'e r\'esultant en la r\'eduction de la fluctuation du faisceau jusqu'a 67~nm, compatible avec la r\'esolution.\par
Deux am\'eliorations ont \'et\'e faites sur le syst\'eme apr\`es ces \'etudes. En premi\`ere lieu, les plans horizontal et vertical pourront \^{e}tre analys\'es simultanement. En second lieu, des filtres furent integr\'es au syst\'eme pour r\'eduire l'effet de la difference de frequence dans le processus de \og~down-mixing~\fg des signaux.
\end{abstract}