\chapter{Purpose of a linear collider}
The physics potential of future linear colliders has been studied since the Standford Linear Collider (SLC)~\cite{Feldman88,SLC91}. The advantage of a linear lepton collider with respect to the LHC is the cleanliness of the events where two elementary particles with known kinematics and spin define the initial state. The resulting precision of the measurements is achievable because of the high resolution possible in the detector due to background processses well calculated and measured, a clean experimental environment, ability to scan systematically in c.o.m energy, possibility of high degree of polarization, and possibility for $\gamma\gamma$, $e^-e^-$, $e^-\gamma$ collisions.\par
\section{Luminosity}
Luminosity, $L$, is proportional to the number of collisions that are produced when two beams cross each other. The expression that relates luminosity, cross section $\sigma$ and the number of events produced $R$ is given by,
\begin{equation}
 R=L\sigma
\end{equation}
Luminosity will depend on the bunch population 	$N$ (assuming an equal number of particles for both beams) and their density distribution within the bunches 

In the Final Focus the goal is to minimize the beam size to recover the luminosity $L$ of a circular collider, limiting the energy loss due to beamstrahlung $\delta_{BS}$. Equation \ref{eq:lum_rad} highlights the dependence with beam size in the tranversal planes.
\begin{equation}
 L \propto \frac{f_{rep}n_b^2}{\sigma_x\sigma_y}\qquad\delta_{BS}\propto\frac{n_b^2E}{(\sigma_x+\sigma_y)^2}\label{eq:lum_rad}
\end{equation}
Table (\ref{t:lum_rad}) shows how the beam size is decreased in all linear collider projects to compensate lower repetition rate and charges. In addition, horizontal beam size is larger than vertical beam size to preserve luminosity while reducing the beam strahlung effect.\par
\begin{table}[!htb]
{\scriptsize
\centering
\begin{tabular}{|l|c||c|c|c|c|}\hline
Parameter & Symbol & LHC & ILC & CLIC 500 GeV& CLIC 3 TeV\\\hline\hline
Energy/z (TeV) & $E$& 7& 0.250 & 0.250 & 1.500\\
Bunch population & $n_b$ &$1.15\times10^{11}$&$2\times10^{10}$&$6.8\times10^9$&$3.72\times10^9$\\
Repetition rate [Hz] &$f_{rep}$& $11.1\times10^{3}$&5 &50&50\\
H/V. IP beam size [nm] & $\sigma_x/\sigma_y$&$16.6\times10^{3}$&474/5.9&202/2.3&40/1\\\hline
E loss (Beamstrahlung) [$\Delta E/E$] &$\delta_{BS}$&-???&0.07&0.07&0.28\\
Luminosity &$L$& $10^{34}$ &$1.57\times10^{34}$ & $2.3\times10^{34}$&$5.9\times10^{34}$\\\hline
\end{tabular}\caption{Luminosity and beamstrahlung for the three current linear collider projects. LHC luminosity is added to compare the beam size and repetition rate from circular to linear colliders.}\label{t:lum_rad}
}
\end{table}


\begin{table}[!htb]
\scriptsize
\centering
\begin{tabular}{|l|c||c|c|c|c|}\hline
Parameter & Symbol & LHC & ILC & CLIC 500 GeV& CLIC 3 TeV\\\hline\hline
Distance from IP to QD0 [m] & $L^*$&23& 3.5/4.5 & 4.3 & 3.5\\
Vertical $\beta$ at the IP [mm] &$\beta_y^*$& 500 & 0.48 & 0.1&0.07\\
Energy Spread [$10^{-3}$]& $\delta$&0.13&3&3&3\\\hline
Vertical Chromaticity & $\xi_y\approx L^*/\beta^*$&46&7300/9400&43000&50000\\\hline
% Luminosity &$L$& $10^{34}$ &$1.57\times10^{34}$ & $2.3\times10^{34}$&$5.9\times10^{34}$\\\hline
\end{tabular}\caption{Approximative vertical chromaticity for the three current linear collider projects. LHC is added for comparison.}\label{t:lum_rad}
\end{table}


