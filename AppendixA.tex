\chapter{The primitive \texorpdfstring{$\mathscr{F}$}{F}}\label{c:primitiveF}

{\scriptsize
 \begin{align*}
  \mathscr{F}&=\frac{1}{1209600.0} \{1323 \cos(5\sqrt{k}L)-675 \cos(7\sqrt{k}L)\\
        &\hspace{1cm}+\sqrt{k}L (378000 \sin(\sqrt{k}L)+21000 \sin(3\sqrt{k}L)-7560 \sin(5\sqrt{k}L))\\
        &\hspace{0.5cm}+\sqrt{k}l^* [23625 \sin(\sqrt{k}L)+4725 \sin(3\sqrt{k}L)-14175 \sin(5\sqrt{k}L)+4725 \sin(7\sqrt{k}L)\\
        &\hspace{1cm}+\sqrt{k}L  (-37800 \cos(5\sqrt{k}L))\\
        &\hspace{1cm}+(\sqrt{k}L)^2 (-75600 \sin(3\sqrt{k}L)+226800 \sin(\sqrt{k}L))]\\
        &\hspace{0.5cm}+(\sqrt{k}l^*)^2 [-49707 \cos(5\sqrt{k}L)+14175 \cos(7\sqrt{k}L)\\
        &\hspace{1cm}+\sqrt{k}L  (1587600 \sin(\sqrt{k}L)-172200 \sin(3\sqrt{k}L)+68040 \sin(5\sqrt{k}L))]\\
        &\hspace{0.5cm}+(\sqrt{k}l^*)^3 [-80325 \sin(\sqrt{k}L)-144725 \sin(3\sqrt{k}L)+82215 \sin(5\sqrt{k}L)-23625 \sin(7\sqrt{k}L)\\
        &\hspace{1cm}+\sqrt{k}L  (37800 \cos(5\sqrt{k}L))\\
        &\hspace{1cm}+(\sqrt{k}L)^2 (680400 \sin(\sqrt{k}L)-126000 \sin(3\sqrt{k}L))]\\
        &\hspace{0.5cm}+(\sqrt{k}l^*)^4 [68985 \cos(5\sqrt{k}L)-23625 \cos(\sqrt{k}L)\\
        &\hspace{1cm}+\sqrt{k}L  (2041200 \sin(\sqrt{k}L)-205800 \sin(3\sqrt{k}L)+37800 \sin(5\sqrt{k}L))]\\
        &\hspace{0.5cm}+(\sqrt{k}l^*)^5 [-458325 \sin(\sqrt{k}L)-43225 \sin(3\sqrt{k}L)-25893 \sin(5\sqrt{k}L)+14175 \sin(7\sqrt{k}L)\\
        &\hspace{1cm}     +\sqrt{k}L  (68040 \cos(5\sqrt{k}L))\\
        &\hspace{1cm}     +(\sqrt{k}L)^2 (680400 \sin(\sqrt{k}L)-25200 \sin(3\sqrt{k}L))]\\
        &\hspace{0.5cm}+(\sqrt{k}l^*)^6 [-945 \cos(5\sqrt{k}L)+4725 \cos(7\sqrt{k}L)\\
        &\hspace{1cm}     +\sqrt{k}L  (831600 \sin(\sqrt{k}L)-12600 \sin(3\sqrt{k}L)-37800 \sin(5\sqrt{k}L))]\\
        &\hspace{0.5cm}+(\sqrt{k}l^*)^7 [(-354375) \sin(\sqrt{k}L) + 5425 \sin(3\sqrt{k}L)-1323 \sin(5\sqrt{k}L)-675 \sin(7\sqrt{k}L)\\
        &\hspace{1cm}+\sqrt{k}L  (-7560 \cos(5\sqrt{k}L))\\
        &\hspace{1cm}+(\sqrt{k}L)^2 (226800 \sin(\sqrt{k}L) + 25200 \sin(3\sqrt{k}L))]\\
        &\hspace{0.5cm}+\cos(\sqrt{k}L) ((\sqrt{k}l^*)^2+1)(4725)[(\sqrt{k}l^*)^5(\sqrt{k}L)(80)\\
        &\hspace{1cm}     +(\sqrt{k}l^*)^4 (155-48(\sqrt{k}L)^2)\\
        &\hspace{1cm}     +(\sqrt{k}l^*)^3 (\sqrt{k}L)(64)\\
        &\hspace{1cm}     +(\sqrt{k}l^*)^2 (182-96 (\sqrt{k}L)^2)\\
        &\hspace{1cm}     +\sqrt{k}l^*  (-16 \sqrt{k}L)\\
        &\hspace{1cm}     +(\sqrt{k}L)^2 (-48)+75]\\
        &\hspace{0.5cm}+\cos(3\sqrt{k}L) (-175) [(\sqrt{k}l^*)^7 (\sqrt{k}L) (120)\\
        &\hspace{1cm}    +(\sqrt{k}l^*)^6(3)(144(\sqrt{k}L)^2+71)\\
        &\hspace{1cm}    +(\sqrt{k}l^*)^5 (\sqrt{k}L) (744)\\
        &\hspace{1cm}    +(\sqrt{k}l^*)^4 (720(\sqrt{k}L)^2+347)\\
        &\hspace{1cm}    +(\sqrt{k}l^*)^3 (\sqrt{k}L)(-24)\\
        &\hspace{1cm}    +(\sqrt{k}l^*)^2 (144 (\sqrt{k}L)^2-473)\\
        &\hspace{1cm}    +\sqrt{k}l^* (-648) \sqrt{k}L\\
        &\hspace{1cm}    +(\sqrt{k}L)^2 (-144)-31]\}
 \end{align*}
}
In order to confirm that the code implementation gave the same result than the original double integral, random values were assigned to $\sqrt{k}l^*$ and $\sqrt{k}L$ and both expressions, the solved and the double integral, were numerically evaluated with difference lower than $10^{-3}$ relative.\par