\begin{abstract}[english]
The exploration of new physics in the TeV scale with precision measurements requires lepton colliders providing high luminosities to obtain enough statistics for the data analysis. In order to achieve design luminosity values, linear colliders feature nanometer beam spot sizes at the Interaction~Point~(IP).\par
In addition to several effects affecting the luminosity, three main issues to achieve the beam size demagnification in the Final Focus Section (FFS) of the accelerator are the chromaticity correction, the synchrotron radiation effects and the correction of the lattice errors.\par
The current linear collider projects, CLIC and ILC, have lattices designed using the local or non-local chromaticity correction schemes. A new chromaticity correction scheme, called non-interleaved, is proposed to the local and non-local chromaticity corrections for CLIC. This lattice is designed and diagnosed, where the main issue in the current state of lattice design is the non-zero second order dispersion in the Final Doublet~(FD) region where a strong sextupole is used to correct the remaining geometrical components. It could be solved by cancelling the second order dispersion and its derivative before the FD.\par
The radiation effect can be evaluated by tracking particles through the lattice or by analytical approximations during the design stage of the lattices. In order to include both, radiation and optic parameters, during the design optimization process, two particular radiation phenomena are reviewed: the Oide effect~\cite{Oide} and the radiation caused by bending magnets~\cite{Sands}.\par
The analytical result of the radiation in bending magnets in~\cite{Sands} was generalized to the case with non-zero alpha and non-zero dispersion at the IP, required during the design and luminosity optimization process. The closed solution for one dipole and one dipole with a drift is compared with the tracking code PLACET~\cite{Placet}, resulting in the improvement of the tracking code results.\par% Finally the model validity for the FFS design is analyzed concluding an agreement within $\pm10\%$ between the theoretical contribution to beam size and the tracking.\par
In the Oide effect, radiation in the final quadrupole sets a limit on the vertical beamsize. Only for CLIC 3~TeV this limit is significant, therefore two possibilities are explored to mitigate its contribution to beam size: double the length and reduce the QD0 gradient, or the integration of a pair of octupoles before and after QD0.\par
% The best result with octupoles demonstrates vertical beam size reduction of $(4.3\pm0.5)$\%, with little or negative impact on luminosity. The correction scheme is currently limited by the phase advance and $\beta_y/\beta_x$ ratio between correctors. It may be possible to improve its performance by slicing QD0.\par
Part of the requirements of the FFS for new linear accelerators, in particular ILC, are tested in The Accelerator Test Facility (ATF). The beam size reduction using the local chromaticity correction is explored by an extension of the original design, called ATF2 with two goals: ({\textbf{goal 1}}) achieve 37~nm of vertical beam size at the IP and ({\textbf{goal 2}}) the stabilization of the IP beam position at the level of few nanometres. Since 2014 beam size of 44~nm are achieved as a regular basis at charges of about~$0.1\times10^{10}$ particules per bunch.\par
A set of three cavities (IPA, IPB and IPC), two upstream and one downstream of the nominal IP, were installed and are used to measure the beam trajectory in the IP region, thus providing enough information to reconstruct the bunch position and angle at the IP. These will be used to for beam stabilization and could detect beam drift/jitter beyond the tolerable margin and undetected optics mismatch affecting the beam size measurements.\par
The specifications required of 1~nm resolution over 10~$\mu$m dynamic range at $1.0\times10^{10}$ particules per bunch with the ATF2 nominal optics have not been yet achieved. The results of the studies in the vertical plane of the cavities calibration show linearity within 5\% over two orders of magnitude of signal attenuation. The minimum resolution achieved is just below 50~nm at~$0.4\times10^{10}$ particules per bunch with a set of electronics impossing a noise limit on resolution of 10~nm per cavity. The dynamic range is 10~$\mu$m at 10~dB attenuation and $0.4\times10^{10}$ particules per bunch, indicating the need to upgrade the electronics. The integration to the ATF tuning instruments is ongoing. Nonetheless, feedback has been tested resulting in reduction of beam jitter down to 67~nm, compatible with resolution.\par
Two improvements have been done on the system since this study. First, the horizontal and vertical planes can be analyzed simultaneouly, such that data can be checked for coupling from one plane to another. Second, filters are added to the system in order to reduce the effect of the mismatch between frequencies in the electronics down-mixing process.\par
\end{abstract}