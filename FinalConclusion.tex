\chapter*{Conclusions, Results and Perspectives}
% \addcontentsline{toc}{chapter}{Final Conclusions}
Linear colliders feature nanometer IP beam spot sizes in order to achieve design luminosities.\par
The CLIC and ILC lattices have been designed using the local or non-local chromaticity correction schemes. A new chromaticity correction scheme is proposed to the local and non-local chromaticity corrections for CLIC~500~GeV. This lattice has been design and diagnose. The main issue in the current state is the non-zero second order dispersion in the FD region where a strong sextupole is used to correct the remaining geometrical components. It could be solved by cancelling the second order dispersion and its derivative before the FD.\par
Radiation effects are crucial during the design stage of the lattices, where effects can be evaluated by tracking particles through the lattice or by analytical approximations. Both, radiation and optic parameters, during the design optimization process, radiation phenomena is reviewed. This document addressed two particular radiation phenomena: the Oide effect \cite{Oide} and the radiation caused by bending magnets \cite{Sands}.\par
In the Oide effect, radiation in the final quadrupole sets a limit on the vertical beamsize. Only for CLIC 3 TeV this limit is significant, therefore two possibilities have been explored to mitigate its contribution to beam size: double the length and reduce the QD0 gradient, or the integration of a pair of octupoles before and after QD0.\par
The best result with octupoles demonstrated vertical beam size reduction of $(4.3\pm0.2)$\%, with little or negative impact on luminosity. The correction scheme is currently limited by the phase advance and $\beta_y/\beta_x$ ratio between correctors. It may be possible to improve its performance by slicing QD0.\par
The radiation in bending magnets has been reviewed. The analytical result in \cite{Sands} was generalized to the case with non-zero alpha at the IP and non-zero dispersion, required during the desing and luminosity optimization process. The closed solution for one dipole and one dipole with a drift was compared with the tracking code PLACET \cite{Placet} resulting in the improvement of the tracking code results. Finally the model validity for the FFS design is analyzed concluding an agreement within $\pm10\%$ between the theoretical contribution to beam size and the tracking.\par

ATF serves as R\&D platform for the requirements of linear accelerators, in particular ILC. The beam size reduction using the local chromaticity correction is explored by an extension of the original design, called ATF2 with two goals: ({\textbf{goal 1}}) achieve 37~nm of vertical beam size at the IP and ({\textbf{goal 2}}) the stabilization of the IP beam position at the level of few nanometres. Since 2014 beam size of 44~nm are achieved as a regular basis at charges of about~$0.1\times10^{10}$ particules per bunch. Possible contributions to beam size are: (1) the increase of the incoming beam emittance along the ATF2 line, (2) systematic errors and resolution limitations on the beam size monitor, (3) beam drift/jitter beyond the tolerable margin and  (4) undetected optics mismatch. Last two issues can be adressed by measuring the beam trajectory in the IP Region after the Final Doublet. In addition, looking forward to \textbf{goal 2}, beam position measurement is a requirement for beam stabilization.\par
Therefore, a set of three cavities (IPA, IPB and IPC), two upstream and one downstream of the nominal IP, were installed and are used to measure the beam trajectory in the IP region, thus providing enough information to reconstruct the bunch position and angle at the IP.\par
The results of the studies in the vertical plane of the cavities calibration show linearity within 5\% over two orders of magnitude of signal attenuation. The minimum resolution achieved is just below 50~nm at~$0.4\times10^{10}$ particules per bunch with a set of electronics impossing a noise limit on resolution of 10~nm per cavity. The dynamic range is 10~$\mu$m at 10~dB attenuation and $0.4\times10^{10}$ particules per bunch, indicating the need to upgrade the electronics. The integration to the ATF tuning instruments is ongoing. Nonetheless, feedback has been tested resulting in reduction of beam jitter down to 67~nm, compatible with resolution.\par
These results are for the moment far from the required specifications with nominal optics of 1~nm resolution over 10~$\mu$m dynamic range at $1.0\times10^{10}$ particules per bunch. Two improvements have been done on the system since this study. First, the horizontal and vertical planes can be analyzed simultaneouly, such that data can be checked for coupling from one plane to another. Second, filters are added to the system in order to reduce the effect of the mismatch between frequencies in the down-mixing process.\par
